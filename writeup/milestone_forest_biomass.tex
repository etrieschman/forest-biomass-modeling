\documentclass{article}

% if you need to pass options to natbib, use, e.g.:
%     \PassOptionsToPackage{numbers, compress}{natbib}
% before loading neurips_2020

% ready for submission
% \usepackage{neurips_2020}

% to compile a preprint version, e.g., for submission to arXiv, add add the
% [preprint] option:
%     \usepackage[preprint]{neurips_2020}

% to compile a camera-ready version, add the [final] option, e.g.:
%     \usepackage[final]{neurips_2020}

% to avoid loading the natbib package, add option nonatbib:
     \usepackage[nonatbib]{neurips_2020}

\usepackage[utf8]{inputenc} % allow utf-8 input
\usepackage[T1]{fontenc}    % use 8-bit T1 fonts
\usepackage{hyperref}       % hyperlinks
\usepackage{url}            % simple URL typesetting
\usepackage{booktabs}       % professional-quality tables
\usepackage{amsfonts}       % blackboard math symbols
\usepackage{nicefrac}       % compact symbols for 1/2, etc.
\usepackage{microtype}      % microtypography

\title{Predicting forest carbon stocks in the Eastern U.S.}

% The \author macro works with any number of authors. There are two commands
% used to separate the names and addresses of multiple authors: \And and \AND.
%
% Using \And between authors leaves it to LaTeX to determine where to break the
% lines. Using \AND forces a line break at that point. So, if LaTeX puts 3 of 4
% authors names on the first line, and the last on the second line, try using
% \AND instead of \And before the third author name.

\author{%
  Erich Trieschman\\
  Department of Statistics, M.S.\\
  Stanford University\\
  Stanford, CA 94305\\
  \texttt{etriesch@stanford.edu} \\
}

\begin{document}

\maketitle

\begin{abstract}
  PLACEHOLDER
\end{abstract}

%------------------------------------------------------------------------
\section{Introduction}


%------------------------------------------------------------------------
\subsection{Problem statement}


%------------------------------------------------------------------------
\subsection{Related work}

Image classification is a common problem solved using convolutional neural networks. [[DESCRIPTIONS OF IMAGENET, ALEXNET, VGGNET, GOOGLENET, AND RESNET]] \cite{VGGNet, GoogLeNET, ResNET}. 

[[PAPER]] Expanded the possibilities of using pretrained models in the described frameworks for use in more targeted settings. [[SUMMARY OF APPROACH]]. [[SUMMARY OF FINDINGS]].

In Feng et al.'s work on long-tailed object detection, they explore training classification models on highly-similar objects, a similar challenge to that conducted in this paper \cite{Feng_2021_ICCV}.  [[SUMMARY OF APPROACH]]. [[SUMMARY OF FINDINGS]]

Carpentier et al. use a self-collected dataset to a very similar problem: tree detection through bark \cite{Carpentier_2018}. [[SUMMARY OF APPROACH]]. [[SUMMARY OF FINDINGS]].

Fricker et al. also attempts to classify trees, however from an aerial perspective instead of the profile perspective taken in this paper \cite{Fricker_RS_2019}. [[SUMMARY OF APPROACH]]. [[SUMMARY OF FINDINGS]]

%------------------------------------------------------------------------
\section{Dataset}


%------------------------------------------------------------------------
\section{Technical approach}


%------------------------------------------------------------------------
\section{Preliminary results}



%------------------------------------------------------------------------
\section*{Broader Impact}



\section*{References}

{\small
\bibliographystyle{ieee}
\bibliography{egbib}
}

\end{document}